\documentclass[11pt]{article}
\renewcommand{\baselinestretch}{1.05}
\usepackage{amsmath,amsthm,verbatim,amssymb,amsfonts,amscd, graphicx}
\usepackage{graphics, multirow}
\topmargin0.0cm
\headheight0.0cm
\headsep0.0cm
\oddsidemargin0.0cm
\textheight23.0cm
\textwidth16.5cm
\footskip1.0cm
\theoremstyle{plain}
\newtheorem{theorem}{Theorem}
\newtheorem{corollary}{Corollary}
\newtheorem{lemma}{Lemma}
\newtheorem{proposition}{Proposition}
\newtheorem*{surfacecor}{Corollary 1}
\newtheorem{conjecture}{Conjecture}
\newtheorem{question}{Question}
\theoremstyle{definition}
\newtheorem{definition}{Definition}
\let\mbb\boldsymbol
\renewcommand\boldsymbol{\mbb}
\renewcommand{\a}{\"{a}}
\renewcommand{\o}{\"{o}}
\renewcommand{\u}{\"{u}}
\newcommand{\beequal}{\mathop{=}\limits^!}
\newcommand{\ray}{\Lambda_A(\mbb{x})}
\newcommand{\dray}{\nabla \Lambda_A(\mbb{x})}

\begin{document}

\title{Numerische Mathematik f\u r Ingenieure (SS 14) - \"{U}bung 6}
\author{Merikan Koyun \& Julian Andrej}
\maketitle

\section*{T10. mehrdimensionales Newton-Verfahren an einem Beispiel}
Gegeben ist folgendes nicht-lineares Gleichungssystem:
\begin{align}
e^{1-x} = \cos(y)-0.2 \\
x^2+y-(1+y)x = \sin(y)+0.2
\end{align}

Um das Gleichungssystem mittels Newton-Verfahren zu L\o sen, stellen wir zun\a chst die Gleichungen um, und erhalten folgende Abbildung:
\begin{equation}
f(x,y) = 
\begin{pmatrix}
\cos(y) - 0.2 - e^{1-x} \\
\sin(y) + 0.2 + (1+y)x - y -x^2
\end{pmatrix}
\end{equation}

Wir berechnen die Jacobi-Matrix:
\begin{equation}
Df(x,y) =
\begin{pmatrix}
\frac{\partial f_1}{\partial x} & \frac{\partial f_1}{\partial y} \\
\frac{\partial f_2}{\partial x} & \frac{\partial f_2}{\partial y}
\end{pmatrix}
=
\begin{pmatrix}
e^{1-x} & -\sin(y) \\ 
1+y-2x & \cos(y)+x-1
\end{pmatrix}
\end{equation}

Die Newton-Iterationsvorschrift lautet:
\begin{equation}
\mbb{x}^{(m+1)} = \mbb{x}^{(m)} - Df(\mbb{x}^{(m)})^{-1} f(\mbb{x}^{(m)})
\end{equation}

Mit dem Startwert $(1, 0)^T$ ergibt sich:
\begin{align*}
\mbb{x}^{(m+1)} &= 
\begin{pmatrix}
1 \\ 0
\end{pmatrix}
- 
\begin{pmatrix}
1 & 0 \\ -1 & 1
\end{pmatrix}
^{-1}
\cdot
\begin{pmatrix}
-0.2 \\ 0.2
\end{pmatrix} \\
&=
\begin{pmatrix}
1 \\ 0
\end{pmatrix}
- 
\begin{pmatrix}
1 & 0 \\ 1 & 1
\end{pmatrix}
\cdot
\begin{pmatrix}
-0.2 \\ 0.2
\end{pmatrix} \\
&=
\begin{pmatrix}
1.2 \\ 0
\end{pmatrix}
\end{align*}





\section*{T11. Rayleigh-Quotient und Eigenwerte}
\begin{itemize}
\item[a)]
Gegeben ist $A \in \mathbb{R}^{n \times n}$ und symmetrisch. Es soll gezeigt werden, dass der Rayleigh-Quotient
\begin{equation}
\Lambda_A(x) = \frac{\langle A\mbb{x},\mbb{x}\rangle_2}{\langle \mbb{x},\mbb{x}\rangle_2} 
\end{equation}
in einem Punkt $\mbb{x}\in \mathbb{R}^n \backslash \{0\}$ durch:
\begin{equation}
\ray = \frac{2(A\mbb{x} - \ray \mbb{x})}{\langle \mbb{x},\mbb{x}\rangle_2}
\end{equation}

Zun\a chst werden die partiellen Ableitungen der Skalarprodukte gebildet:
\begin{align}
\langle A\mbb{x},\mbb{x}\rangle_2 &= \mbb{x}^TA\mbb{x}  = (A+A^*)\mbb{x} = 2A\mbb{x} \\
\langle \mbb{x},\mbb{x}\rangle_2 &= \mbb{x}^T\mbb{x} = 2\mbb{x} 
\end{align}

Mit Hilfe der Quotientenregel ergibt sich:
\begin{align}
\dray 
&= \dfrac{\nabla(\mbb{x}^TA\mbb{x}) (\mbb{x}^T\mbb{x}) - (\mbb{x}^TA\mbb{x}) \nabla(\mbb{x}^T\mbb{x})}{\mbb{x}^T\mbb{x}\mbb{x}^T\mbb{x}} \\
&= \dfrac{2A\mbb{x}(\mbb{x}^T\mbb{x}) - (\mbb{x}^TA\mbb{x})2\mbb{x} }{\mbb{x}^T\mbb{x}\mbb{x}^T\mbb{x}} \\
&= \dfrac{2A\mbb{x} - 2\dfrac{\mbb{x}^TA\mbb{x}}{\mbb{x}^T\mbb{x}}\mbb{x}}{\mbb{x}^T\mbb{x}} \\ 
&= \dfrac{2(A\mbb{x} - \ray \mbb{x})}{\langle \mbb{x},\mbb{x}\rangle_2} _\blacksquare
\end{align}

\item[b)]
Gegeben ist $A \in \mathbb{R}^{n \times n}$ und symmetrisch und $\mbb{x}\in \mathbb{R}^n \backslash \{0\}$ ein Punkt, in dem der Rayleigh Quotient $\Lambda_A$ ein Maximum besitzt. Es soll gezeigt werden, dass $\Lambda_A(\mbb{x})$ ein Eigenwert zum Eigenvektor $\mbb{x}$ ist, also $A\mbb{x} = \ray \mbb{x}$ gilt.

F\u r das Maximum gilt, dass
\begin{equation}
\dray = \frac{2(A\mbb{x} - \ray \mbb{x})}{\langle \mbb{x},\mbb{x}\rangle_2} = 0
\label{max}
\end{equation}

Damit \eqref{max} erf\u llt werden kann, muss der Z\a hler Null werden. Es folgt:
\begin{align}
2(A\mbb{x} - \ray \mbb{x}) &\beequal 0 \\
A\mbb{x} - \ray \mbb{x} &= 0\\
A\mbb{x} &= \ray \mbb{x} \qquad_\blacksquare
\end{align}

\end{itemize}
\end{document}