\documentclass[11pt]{article}
\renewcommand{\baselinestretch}{1.05}
\usepackage{amsmath,amsthm,verbatim,amssymb,amsfonts,amscd, graphicx}
\usepackage{graphics, multirow}
\topmargin0.0cm
\headheight0.0cm
\headsep0.0cm
\oddsidemargin0.0cm
\textheight23.0cm
\textwidth16.5cm
\footskip1.0cm
\parindent0pt
\theoremstyle{plain}
\newtheorem{theorem}{Theorem}
\newtheorem{corollary}{Corollary}
\newtheorem{lemma}{Lemma}
\newtheorem{proposition}{Proposition}
\newtheorem*{surfacecor}{Corollary 1}
\newtheorem{conjecture}{Conjecture}
\newtheorem{question}{Question}
\theoremstyle{definition}
\newtheorem{definition}{Definition}
\let\mbb\boldsymbol
\renewcommand\boldsymbol{\mbb}
\renewcommand{\a}{\"{a}}
\renewcommand{\o}{\"{o}}
\renewcommand{\u}{\"{u}}
\newcommand{\beequal}{\mathop{=}\limits^!}

\begin{document}

\title{Numerische Mathematik f\u r Ingenieure (SS 14) - \"{U}bung 3}
\author{Merikan Koyun \& Julian Andrej}
\maketitle

\section*{T4. Spaltensummennorm}
Es soll gezeigt werden, dass $\Vert A\Vert_1 = \max_{j=1,...,n} \sum_{i=1}^n |a_{ij}|$ gilt, wobei $\Vert\cdot\Vert_1$ die induzierte Matrixnorm ist. Es gilt ausserdem $\Vert x \Vert_1 = \sum_{i=1}^n |x_i|$. \\
Zun\a chst wird gezeigt, dass gilt
\begin{equation}
\Vert Ax\Vert_1 \leq m \Vert x\Vert_1, \quad \forall x \in \mathbb{R}^n \backslash \{0\}, \quad m=\Vert A \Vert_1
\label{beh}
\end{equation}
Wir beginnen mit der linken Seite der Ungleichung. Da das Produkt $Ax$ einem Vektor entspricht, so muss auch $\Vert\cdot \Vert_1$ f\u r $Ax$ definiert sein. Wir definieren uns die $i$-te Zeile der Matrix $A$ als $a_{i*}$.

Die Ungleichung l\a sst sich also schreiben als:
\begin{align*}
\Vert Ax \Vert_1 =  \sum_{i=1}^n |a_{i*}x| = \sum_{i=1}^n \left| \sum_{j=1}^n a_{ij}x_i \right| 
 \,\,\leq\,\, m \sum_{j=1}^n |x_j| = \max_{j=1,...,n} \sum_{i=1}^n |a_{ij}| \sum_{j=1}^n |x_j| \end{align*}
Trivialerweise muss gelten, dass:
\begin{equation}
 \sum_{i=1}^n \sum_{j=1}^n |a_{ij}||x_j| \leq \max_{j=1,...,n} \sum_{i=1}^n |a_{ij}| \sum_{j=1}^n |x_j|
\end{equation}
Es folgt die Ungleichung:
\begin{equation}
\sum_{i=1}^n \left| \sum_{j=1}^n a_{ij}x_i \right|  \leq  \sum_{i=1}^n \sum_{j=1}^n |a_{ij}||x_j|
\end{equation}
Es l\a sst sich unmittelbar erkennen, dass die linke Ungleichung zu einer Gleichheit wird f\u r $a_{ij} \geq 0$ und $x_i \geq 0,\, \forall i,j \in \{1,...,n\}$, da die Betr\a ge gestrichen werden k\o nnen. Sobald $a_{ij} < 0$ oder $x_i < 0$ f\u r ein beliebiges $i,j \in \{1,...,n\}$ gilt die Ungleichung.\\

Um nun die R\u ckrichtung zu beweisen gen\u gt es einen Vektor zu finden, der \eqref{beh} mit Gleichheit erf\u llt, da dann die Ungleichheit aus \eqref{beh} wegf\a llt. Wir w\a hlen $x=e_k$ und w\a hlen $k$ so, dass $\sum_{i=1}^n |a_{ik}|$ maximal wird, wobei $e_k \in \mathbb{R}^n$ der $k$-te Einheitsvektor ist. Es gilt offensichtlich $\Vert e_k \Vert_1 = 1$. Somit l\a sst sich die linke Seite von \eqref{beh} schreiben als:
\begin{equation}
\Vert Ae_k \Vert_1 = \Vert A_{*k} \Vert_1
\end{equation}
Wir erhalten also eine Spalte aus der Matrix $A$, die maximal ist. F\u r die Spalte l\a sst sich wiederum $\Vert\cdot\Vert_1$ anwenden, da wir einen Spaltenvektor haben. Es gilt also:
\begin{equation}
\Vert A_{*k} \Vert_1 = \sum_{i=1}^n |a_{ik}|
\end{equation}
woraus sich mit $\Vert e_k\Vert_1 = 1$ f\u r \eqref{beh} Gleichheit ergibt, da wir $k$ so gew\a hlt haben, dass die Spalte maximal ist:
\begin{equation}
\sum_{i=1}^n |a_{ik}| = \max_{j=1,...,n} \sum_{i=1}^n |a_{ij}|.
\end{equation}
Somit ist die Behauptung bewiesen. $_\blacksquare$


\section*{T5. QR-Zerlegung mit Givens-Rotationen an einem Beispiel}
\begin{itemize}
\item[a)]
Gegeben ist die Matrix
\begin{equation}
A = 
\begin{pmatrix}3 & -9 & 7\cr -4 & -13 & -1\cr 0 & -20 & -35\end{pmatrix}
\end{equation}
Wir wollen den Eintrag $a_{21} = -4$ Null setzen. Dazu wenden wir eine Givens-Rotation $Q_{12}$ auf $A$ an. Wir w\a hlen $a=3$ und $b=-4$. Wir berechnen uns die Eintr\a ge von $Q$:
\begin{align*}
c = \dfrac{a}{\sqrt{a^2+b^2}} \\
s = \dfrac{b}{\sqrt{a^2+b^2}}
\end{align*}
$Q$ ist definiert als:
\begin{equation}
Q=
\begin{pmatrix}
c & s \\ 
-s & c
\end{pmatrix}
\end{equation}
Wir wollen $Q_{12}$ bestimmen: 
\begin{equation}
Q_{12}=
\begin{pmatrix}\frac{3}{5} & -\frac{4}{5} & 0\cr \frac{4}{5} & \frac{3}{5} & 0\cr 0 & 0 & 1\end{pmatrix}
\end{equation}
Nun berechnen wir:
\begin{equation}
A'=Q_{12}A = 
\begin{pmatrix}5 & 5 & 5\cr 0 & -15 & 5\cr 0 & -20 & -35\end{pmatrix}
\end{equation}

Da der Eintrag $a'_{31}=0$ ist, k\o nnen wir diesen \u berspringen und den Eintrag $a'_{32}=-20$ Null setzen. Dazu wenden wir die Givens-Rotation $Q_{23}$ an.
Wir setzen $a=-15$ und $b=-20$ und berechnen $Q_{23}$ analog zu oben:
\begin{equation}
Q_{23} = 
\begin{pmatrix}1 & 0 & 0\cr 0 & -\frac{3}{5} & -\frac{4}{5}\cr 0 & \frac{4}{5} & -\frac{3}{5}\end{pmatrix}
\end{equation}
Anschliessend wird die Rotation auf $A'$ angewendet und ergibt:
\begin{equation}
R=Q_{23} A' = 
\begin{pmatrix}5 & 5 & 5\cr 0 & 25 & 25\cr 0 & 0 & 25\end{pmatrix}
\end{equation}
$Q$ l\a sst sich nun einfach folgendermassen berechnen:
\begin{equation}
Q = (Q_{23}Q_{12})^T = 
\begin{pmatrix}\frac{3}{5} & -\frac{12}{25} & \frac{16}{25}\cr -\frac{4}{5} & -\frac{9}{25} & \frac{12}{25}\cr 0 & -\frac{4}{5} & -\frac{3}{5}\end{pmatrix}
\end{equation}
Zur Probe wird $QR$ gerechnet:
\begin{equation}
QR = 
\begin{pmatrix}3 & -9 & 7\cr -4 & -13 & -1\cr 0 & -20 & -35\end{pmatrix}=\begin{pmatrix}3 & -9 & 7\cr -4 & -13 & -1\cr 0 & -20 & -35\end{pmatrix}
\end{equation}
Dies entspricht $A$ und somit ist die QR-Zerlegung richtig.


\item[b)]
Gegeben ist das Gleichungssystem $Ax=b$, mit 
\begin{equation}
b=
\begin{pmatrix}1\cr 2\cr 3\end{pmatrix}
\end{equation}
Es wird zun\a chst $z$ bestimmt.
\begin{equation}
z=Q^Tb = 
\begin{pmatrix}-1\cr -\frac{18}{5}\cr -\frac{1}{5}\end{pmatrix}
\end{equation}
Nun kann $Rx=z$ mittels R\u ckw\a rtseinsetzen gel\o st werden:
\begin{align*}
25 x_3 = -\frac{1}{5} \rightarrow x_3 = -\frac{1}{125} \\
25 x_2 + 25 x_3 = -\frac{18}{5} \rightarrow x_2 = -\frac{17}{125} \\
5 x_1 + 5 x_2 + 5 x_3 = -1 \rightarrow x_1 = -\frac{7}{125} \\
\end{align*}
\end{itemize}

\end{document}