\documentclass[11pt]{article}
\renewcommand{\baselinestretch}{1.05}
\usepackage{amsmath,amsthm,verbatim,amssymb,amsfonts,amscd, graphicx}
\usepackage{graphics}
\topmargin0.0cm
\headheight0.0cm
\headsep0.0cm
\oddsidemargin0.0cm
\textheight23.0cm
\textwidth16.5cm
\footskip1.0cm
\parindent0pt
\theoremstyle{plain}
\newtheorem{theorem}{Theorem}
\newtheorem{corollary}{Corollary}
\newtheorem{lemma}{Lemma}
\newtheorem{proposition}{Proposition}
\newtheorem*{surfacecor}{Corollary 1}
\newtheorem{conjecture}{Conjecture}
\newtheorem{question}{Question}
\theoremstyle{definition}
\newtheorem{definition}{Definition}
\let\mbb\boldsymbol
\renewcommand\boldsymbol{\mbb}
\renewcommand{\a}{\"{a}}
\renewcommand{\o}{\"{o}}
\renewcommand{\u}{\"{u}}
\newcommand{\beequal}{\mathop{=}\limits^!}

\begin{document}

\title{Numerische Mathematik f\u r Ingenieure (SS 14) - \"{U}bung 4}
\author{Merikan Koyun \& Julian Andrej}
\maketitle

\section*{T6. Householdermatrizen}
\begin{itemize}
\item[a)]
F\u r $v \in \mathbb{R}^n \backslash \{0\}$ ist $H \in \mathbb{R}^{n\times n}$ definiert:
\begin{equation}
H = I - \frac{2}{v^*v} vv^*
\end{equation}

Bei Orthogonalit\a t gilt $HH^*=I$:
\begin{align}
HH^* &= \left(I-\frac{2}{v^*v}vv^*\right)\left(I-\frac{2}{v^*v}vv^*\right) \\
&=  I - \frac{4}{v^*v}vv^* + \frac{4}{(v^*v)^2} vv^*vv^* \\
&= I -\frac{4}{v^*v}vv^* + \frac{4}{v^*v}vv^* = I
\end{align}

\item[b)]
Nach Anwendung der Vorgaben aus b) gilt:
\begin{equation}
  H = I - 2vv^*
\end{equation}
\begin{align}
  Hx &= x - 2 v \langle v,x \rangle \\
     &= x - 2(x - ke_1) \frac{\|x\|^2 - k\langle e_1, x \rangle}{\|x\|^2 - 2k \langle x, e_1\rangle + k^2 \| e_1 \|^2} \\
     &= x - 2(x-ke_1) \frac{k^2 - k \langle e_1, x \rangle}{2(k^2 - k \langle x, e_1 \rangle)} \\
     &= ke_1
\end{align}

\item[c)]
Die Householder Matrix $H$ ist, wie oben bewiesen, orthogonal. Dadurch erf\u llt sie eine der Bedingungen f\u r eine QR Zerlegung, da zur Anwendung eben dieser, orthogonale Matrizen ben\o tigt werden.

Die Anwendung der Householdermatrix auf eine Matrix $A$ f\u hrt eine Spiegelung der Eintr\a ge $a_{ij}$ an einer Hyperebene, beschrieben durch $v$, durch. Die sukzessive Anwendung der Householdermatrix auf $A$ kann zur Bildung einer oberen-rechten Dreiecksmatrix $R$ verwendet werden, \a hnlich, wie bei der QR-Zerlegung mittels Givens-Rotationen. 

Der elementare Unterschied zur QR-Zerlegung mit Givens-Rotation, besteht darin, dass die Householder Transformation pro Anwendung auf $A$ eine gesamte (Unter-)Spalte eliminiert; im Gegensatz zu einzelnen Eintr\a gen.
\end{itemize}

\section*{T7. Lineares Ausgleichsproblem}
\begin{itemize}
\item[a)]
Das lineare Ausgleichsproblem lautet
\begin{equation}
\min = \| \begin{bmatrix} y_1 \\ \vdots \\ y_n \end{bmatrix} -  \begin{bmatrix} 1 & x_1^2 \\ \vdots & \vdots \\ 1 & x_n^2  \end{bmatrix} \begin{bmatrix} a \\ b \end{bmatrix} \|.
\end{equation}

Daraus folgt die Normalengleichung:

\begin{equation}
\begin{bmatrix} n & x_1^2 + \dots + x_n^2 \\
  x_1^2 + \dots + x_n^2 & x_1^4 + \dots + x_n^4
\end{bmatrix} \begin{bmatrix} a \\ b \end{bmatrix}
= \begin{bmatrix} 1 & \dots & 1 \\ x_1^2 & \dots & x_n^2 \end{bmatrix} \begin{bmatrix} y_1 \\ \vdots \\ y_n \end{bmatrix}.
\end{equation}

\item[b)]
Aus dem Zahlenbeispiel ergibt sich fuer die Normalengleichung:

\begin{equation}
\begin{bmatrix} 30b + 4a \\ 354b + 30a \end{bmatrix} = \begin{bmatrix} 31 \\ 375 \end{bmatrix}
\end{equation}
die Loesung dieses Systems beschreibt die Werte $a$ und $b$.
\begin{align}
  a &= -\frac{23}{43} \approx -0.5348837 \\
  b &= \frac{95}{96} \approx 1.1046511
\end{align}
\end{itemize}

\end{document}
