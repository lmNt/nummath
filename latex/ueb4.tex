\documentclass[11pt]{article}
\renewcommand{\baselinestretch}{1.05}
\usepackage{amsmath,amsthm,verbatim,amssymb,amsfonts,amscd, graphicx}
\usepackage{graphics}
\topmargin0.0cm
\headheight0.0cm
\headsep0.0cm
\oddsidemargin0.0cm
\textheight23.0cm
\textwidth16.5cm
\footskip1.0cm
\parindent0pt
\theoremstyle{plain}
\newtheorem{theorem}{Theorem}
\newtheorem{corollary}{Corollary}
\newtheorem{lemma}{Lemma}
\newtheorem{proposition}{Proposition}
\newtheorem*{surfacecor}{Corollary 1}
\newtheorem{conjecture}{Conjecture}
\newtheorem{question}{Question}
\theoremstyle{definition}
\newtheorem{definition}{Definition}
\let\mbb\boldsymbol
\renewcommand\boldsymbol{\mbb}
\renewcommand{\a}{\"{a}}
\renewcommand{\o}{\"{o}}
\renewcommand{\u}{\"{u}}
\newcommand{\beequal}{\mathop{=}\limits^!}

\begin{document}

\title{Numerische Mathematik f\u r Ingenieure (SS 14) - \"{U}bung 4}
\author{Merikan Koyun \& Julian Andrej}
\maketitle

\section*{T6. Householdermatrizen}
F\u r $v \in \mathbb{R}^n \backslash \{0\}$ ist $H \in \mathbb{R}^{n\times n}$ definiert:
\begin{equation}
H = I - \frac{2}{v^*v} vv^*
\end{equation}

Bei Orthogonalit\a t gilt $HH^*=I$. Ausserdem gilt $v^*v=1$:
\begin{align}
HH^* &= (I-\frac{2}{v^*v}vv^*)(I-\frac{2}{v^*v}vv^*) \\
&=  I - \frac{4}{v^*v}vv^* + \frac{4}{v^*v v^*v} vv^*vv^* \\
&= I - 4vv^* + 4v(v^*v)v^* = I-4vv^* + 4vv^* = I
\end{align}

\section*{T7. Lineares Ausgleichsproblem}

Das lineare ausgleichsproblem lautet
\begin{equation}
\min = \| \begin{bmatrix} y_1 \\ \vdots \\ y_n \end{bmatrix} -  \begin{bmatrix} 1 & x_1^2 \\ \vdots & \vdots \\ 1 & x_n^2  \end{bmatrix} \begin{bmatrix} a \\ b \end{bmatrix} \|.
\end{equation}

Daraus folgt die Normalengleichung:

\begin{equation}
\begin{bmatrix} n & x_1^2 + \dots + x_n^2 \\
  x_1^2 + \dots + x_n^2 & x_1^4 + \dots + x_n^4
\end{bmatrix} \begin{bmatrix} a \\ b \end{bmatrix}
= \begin{bmatrix} 1 & \dots & 1 \\ x_1^2 & \dots & x_n^2 \end{bmatrix} \begin{bmatrix} y_1 \\ \vdots \\ y_n \end{bmatrix}.
\end{equation}

Aus dem Zahlenbeispiel ergibt sich fuer die Normalengleichung:

\begin{equation}
\begin{bmatrix} 30b + 4a \\ 354b + 30a \end{bmatrix} = \begin{bmatrix} 31 \\ 375 \end{bmatrix}
\end{equation}
die Loesung dieses Systems beschreibt die Werte $a$ und $b$.
\begin{align}
  a &= -\frac{23}{43} \approx -0.5348837 \\
  b &= \frac{95}{96} \approx 1.1046511
\end{align}

\end{document}
