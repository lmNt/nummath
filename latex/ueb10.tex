\documentclass[11pt]{article}
\renewcommand{\baselinestretch}{1.05}
\usepackage{amsmath,amsthm,verbatim,amssymb,amsfonts,amscd, graphicx}
\usepackage{graphics, multirow}
\topmargin0.0cm
\headheight0.0cm
\headsep0.0cm
\oddsidemargin0.0cm
\textheight23.0cm
\textwidth16.5cm
\footskip1.0cm
\parindent0cm
\theoremstyle{plain}
\newtheorem{theorem}{Theorem}
\newtheorem{corollary}{Corollary}
\newtheorem{lemma}{Lemma}
\newtheorem{proposition}{Proposition}
\newtheorem*{surfacecor}{Corollary 1}
\newtheorem{conjecture}{Conjecture}
\newtheorem{question}{Question}
\theoremstyle{definition}
\newtheorem{definition}{Definition}
\let\mbb\boldsymbol
\renewcommand\boldsymbol{\mbb}
\renewcommand{\a}{\"{a}}
\renewcommand{\o}{\"{o}}
\renewcommand{\u}{\"{u}}
\newcommand{\beequal}{\mathop{=}\limits^!}
\newcommand{\equalstar}{\mathop{=}\limits^{(1)}}
\newcommand{\equalsstar}{\mathop{=}\limits^{(2)}}
\newcommand{\equalssstar}{\mathop{=}\limits^{(3)}}
\newcommand{\rayx}{\Lambda_A(\mbb{x})}
\newcommand{\rayv}{\Lambda_A(\mbb{v})}
\newcommand{\dray}{\nabla \Lambda_A(\mbb{x})}
\newcommand{\nat}{\mathbb{N}}
\newcommand{\real}{\mathbb{R}}
\newcommand{\reall}[2]{\mathbb{R}^{(#1 \times #2)}}

\begin{document}

\title{Numerische Mathematik f\u r Ingenieure (SS 14) - \"{U}bung 10}
\author{Merikan Koyun \& Julian Andrej}
\maketitle

\section*{T18. $\frac{3}{8}$-Regel}
Mit $l_i$ aus dem Skript (5.2) und
\[ w_i := \int_a^b l_i(x) dx \]
erhalten wir
\[ w_i = \int_a^b \prod\limits_{\substack{k=0 \\ k\neq i}}^m \frac{x-x_j}{x_i-x_j} dx \]
Durch Substitution von $x = a + sh$ und $s \in [0,m]$ ergibt sich
\[ w_i = \int_0^m \prod\limits_{\substack{k=0 \\ k\neq i}}^m \frac{s-j}{i-j} ds \]

Damit k\o nnen wir die Gewichte fuer $m = 3$ bestimmen. F\u r $w_{im}$ ergibt sich damit:
\[ w_{03} = \frac{1}{3} \int_0^3 \frac{x-1}{0-1} \frac{x-2}{0-2} \frac{x-3}{0-3} dx = \frac{1}{8} \]
\[ w_{13} = \frac{1}{3} \int_0^3 \frac{x-0}{1-0} \frac{x-2}{1-2} \frac{x-3}{1-3} dx = \frac{3}{8} \]
\[ w_{23} = \frac{1}{3} \int_0^3 \frac{x-0}{2-0} \frac{x-1}{2-1} \frac{x-3}{2-3} dx = \frac{3}{8} \]
\[ w_{33} = \frac{1}{3} \int_0^3 \frac{x-0}{3-0} \frac{x-1}{3-1} \frac{x-2}{3-2} dx = \frac{1}{8} \]

\section*{T19. summierte Trapezregel}
\begin{itemize}
\item[a)]
Mit den Voraussetzungen aus der Aufgabe sowie der Beschreibung der \a quidstanten Unterteilung in $l\in N$ Teilintervalle
auf S.93 im Skript definieren wir 

\[\mathbf{Q}_{[a,b],l}(f) := \sum_{i=1}^l \mathbf{Q}_{[y_{i-1}, y_i]}(f) \mbox{ } \forall f \in C[a,b] \]
Ausserdem ist aus dem Skript bekannt, dass mit $h = (b-a)/l$ gilt

\[ \int_a^b f(x) dx = \sum_{i=1}^l \int_{y_{i-1}}^{y_i} f(x) dx \approx \frac{h}{2} \sum_{i=1}^l (f(y_{i-1}) + f(y_i)) \]

dies entspricht der summierten Trapezregel $T(h)$.

Berechnet man die Summe erhalten wir:

\[ T(h) = \frac{h}{2} \left[ f(a) + 2 \left( f(x_1) + f(x_2) + \dots + f(x_{l-1}) \right) + f(b) \right] \]

Fassen wir die Terme zusammen ergibt sich

\[ T(h) = h \left[ \frac{f(a)}{2} + \sum_{i=1}^{l-1} f(a+ih) + \frac{f(b)}{2} \right] \]


\item[b)]
Es soll gezeigt werden, dass $T(h)$ f\u r die Bestimmung von $T(\frac{h}{2})$ genutzt werden kann.\\
F\u r $T(\frac{h}{2})$ gilt:

\begin{equation}
T\left(\frac{h}{2}\right) = \frac{h}{2}\left[\frac{f(a)}{2}+\sum_{i=1}^{2l-1} f\left(a+i\frac{h}{2}\right) + \frac{f(b)}{2}\right]
\end{equation}

Die Summe l\a sst sich in zwei Untersummen aufteilen:
\begin{equation}
T\left(\frac{h}{2}\right) = \frac{h}{2}\left[\frac{f(a)}{2}+\sum_{i=1}^{l-1} f\left(a+(2i)\frac{h}{2}\right) +\sum_{i=0}^{l-1} f\left(a+(2i+1)\frac{h}{2}\right) + \frac{f(b)}{2}\right]
\end{equation}

Nun erkennt man, dass die erste Summe der Summe von $T(h)$ entspricht. Man kann also die zweite Summe herausziehen:
\begin{align}
T\left(\frac{h}{2}\right) &= \frac{1}{2}\underbrace{h\left[\frac{f(a)}{2}+\sum_{i=1}^{l-1} f\left(a+ih\right)  + \frac{f(b)}{2}\right]}_{T(h)} + \frac{h}{2}\sum_{i=0}^{l-1} f\left(a+(2i+1)\frac{h}{2}\right) \\
&= \frac{1}{2}T(h) + \frac{h}{2}\sum_{i=0}^{l-1} f\left(a+(2i+1)\frac{h}{2}\right)
\end{align}


\end{itemize}
\end{document}
