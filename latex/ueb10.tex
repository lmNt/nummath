\documentclass[a4paper,12pt]{article}
\usepackage{amsmath}

\begin{document}

\section{T18}
Mit $l_i$ aus dem Skript (5.2) und
\[ w_i := \int_a^b l_i(x) dx \]
erhalten wir
\[ w_i = \int_a^b \prod\limits_{\substack{k=0 \\ k\neq i}}^m \frac{x-x_j}{x_i-x_j} dx \]
Durch substition von $x = a + sh$ und $s \in [0,m]$ ergibt sich
\[ w_i = \int_0^m \prod\limits_{\substack{=0 \\ k\neq i}}^m \frac{s-j}{i-j} ds \]

Damit koennen wir die Gewichte fuer $m = 3$ bestimmen. Fuer $w_{im}$ ergibt sich damit:
\[ w_{03} = \frac{1}{3} \int_0^3 \frac{x-1}{0-1} \frac{x-2}{0-2} \frac{x-3}{0-3} dx = \frac{1}{8} \]
\[ w_{13} = \frac{1}{3} \int_0^3 \frac{x-0}{1-0} \frac{x-2}{1-2} \frac{x-3}{1-3} dx = \frac{3}{8} \]
\[ w_{23} = \frac{1}{3} \int_0^3 \frac{x-0}{2-0} \frac{x-1}{2-1} \frac{x-3}{2-3} dx = \frac{3}{8} \]
\[ w_{33} = \frac{1}{3} \int_0^3 \frac{x-0}{3-0} \frac{x-1}{3-1} \frac{x-2}{3-2} dx = \frac{1}{8} \]

\section{T19}

Mit den Vorraussetzungen aus der Aufgabe sowie der Beschreibung der aequidstanten Unterteilung in $l\in N$ Teilintervalle
auf S.93 im Skript definieren wir 

\[\mathbf{Q}_{[a,b],l}(f) := \sum_{i=1}^l \mathbf{Q}_{[y_{i-1}, y_i]}(f) \mbox{ } \forall f \in C[a,b] \]
Ausserdem ist aus dem Skript bekannt, dass mit $h = (b-a)/l$ gilt

\[ \int_a^b f(x) dx = \sum_{i=1}^l \int_{y_{i-1}}^{y_i} f(x) dx \approx \frac{h}{2} \sum_{i=1}^l (f(y_{i-1}) + f(y_i)) \]

dies entspricht der summierten Trapezregel $T(h)$.

Berechnet man die Summe erhalten wir:

\[ T(h) = \frac{h}{2} \left[ f(a) + 2 \left( f(x_1) + f(x_2) + \dots + f(x_{l-1}) \right) + f(b) \right] \]

Fassen wir die Terme zusammen ergibt sich

\[ T(h) = h \left[ \frac{f(a)}{2} + \sum_{i=1}^{l-1} f(a+ih) + \frac{f(b)}{2} \right] \]


\end{document}
