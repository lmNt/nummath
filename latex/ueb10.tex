\documentclass[a4paper,12pt]{article}
\usepackage{amsmath}

\begin{document}

Mit $l_i$ aus dem Skript (5.2) und
\[ w_i := \int_a^b l_i(x) dx \]
erhalten wir
\[ w_i = \int_a^b \prod\limits_{\substack{k=0 \\ k\neq i}}^m \frac{x-x_j}{x_i-x_j} dx \]
Durch substition von $x = a + sh$ und $s \in [0,m]$ ergibt sich
\[ w_i = \int_0^m \prod\limits_{\substack{=0 \\ k\neq i}}^m \frac{s-j}{i-j} ds \]

Damit koennen wir die Gewichte fuer $m = 3$ bestimmen. Fuer $w_{im}$ ergibt sich damit:
\[ w_{03} = \frac{1}{3} \int_0^3 \frac{x-1}{0-1} \frac{x-2}{0-2} \frac{x-3}{0-3} dx = \frac{1}{8} \]
\[ w_{13} = \frac{1}{3} \int_0^3 \frac{x-0}{1-0} \frac{x-2}{1-2} \frac{x-3}{1-3} dx = \frac{3}{8} \]

w23 und w33 to goooooooooo.......

\end{document}
