\documentclass[11pt]{article}
\renewcommand{\baselinestretch}{1.05}
\usepackage{amsmath,amsthm,verbatim,amssymb,amsfonts,amscd, graphicx}
\usepackage{graphics, multirow}
\topmargin0.0cm
\headheight0.0cm
\headsep0.0cm
\oddsidemargin0.0cm
\textheight23.0cm
\textwidth16.5cm
\footskip1.0cm
\parindent0cm
\theoremstyle{plain}
\newtheorem{theorem}{Theorem}
\newtheorem{corollary}{Corollary}
\newtheorem{lemma}{Lemma}
\newtheorem{proposition}{Proposition}
\newtheorem*{surfacecor}{Corollary 1}
\newtheorem{conjecture}{Conjecture}
\newtheorem{question}{Question}
\theoremstyle{definition}
\newtheorem{definition}{Definition}
\let\mbb\boldsymbol
\renewcommand\boldsymbol{\mbb}
\renewcommand{\a}{\"{a}}
\renewcommand{\o}{\"{o}}
\renewcommand{\u}{\"{u}}
\newcommand{\beequal}{\mathop{=}\limits^!}
\newcommand{\equalstar}{\mathop{=}\limits^{(1)}}
\newcommand{\equalsstar}{\mathop{=}\limits^{(2)}}
\newcommand{\equalssstar}{\mathop{=}\limits^{(3)}}
\newcommand{\ray}{\Lambda_A(\mbb{x})}
\newcommand{\dray}{\nabla \Lambda_A(\mbb{x})}

\begin{document}

\title{Numerische Mathematik f\u r Ingenieure (SS 14) - \"{U}bung 7}
\author{Merikan Koyun \& Julian Andrej}
\maketitle

\section*{T12. Eigenwerte und Eigenvektoren des eindimensionalen Modellproblems}
Es seien $A \in \mathbb{R}^{n \times n}$, $1 \leq n \in \mathbb{N}, h := \frac{1}{n+1}$:
\begin{equation}
A=h^{-2}
\begin{pmatrix}
2 & -1 & & \\
-1 & \ddots & \ddots & \\
 & \ddots & \ddots & -1 \\
 & & -1 & 2 
\end{pmatrix}
\end{equation}
Definiere den Vektor $e^k \in \mathbb{R}^n$
\begin{equation}
e^k_j := \sin(\pi jkh) \text{ f\u r alle } j,k\in \{1,...,n\}
\end{equation}
und
\begin{equation}
\lambda_k := 4h^{-2} \sin^2(\pi kh/2) \text{ f\u r alle } k\in \{1,...,n\} .
\end{equation}
Zeige, dass
\begin{equation}
Ae^k = \lambda_ke^k
\end{equation}
gilt.\vspace{0.2cm}

Wir betrachten eine beliebige $j$-te Komponente von $(Ae^k)_j$. Mit
\begin{equation}
A_{ij} =
 \begin{cases}
   2h^{-2}  & \text{falls } i=j \\
   -h^{-2}  & \text{falls } |i-j|=1 \\
   0 & \text{sonst }
  \end{cases}
\end{equation}
ergibt sich:
\begin{align*}
(Ae^k)_j 
&= h^{-2} ( 2e^k_j - e^k_{j-1} - e^k_{j+1} ) \\
&= h^{-2} ( 2\sin(\pi jkh) - \sin(\pi (j-1)kh) - \sin(\pi (j+1)kh) \\
&= h^{-2} ( 2\sin(\pi jkh) - \sin(\pi jkh - \pi kh) - \sin(\pi jkh + \pi kh) \\
&\equalstar h^{-2} ( 2\sin(\pi jkh) - \sin(\pi jkh)\cos(- \pi kh) -\cos(\pi jkh)\sin(-\pi kh) \\ & \hspace{0.3cm}- \sin(\pi jkh)\cos(\pi kh) - \cos(\pi jkh)\sin(\pi kh)) \\
&\equalsstar h^{-2} ( 2\sin(\pi jkh) - 2\sin(\pi jkh)\cos(\pi kh)) \\
&= h^{-2}\sin(\pi jkh) ( 2 - 2\cos(\pi kh)) \\
&\equalssstar h^{-2}\sin(\pi jkh)  4\sin^2(\pi kh/2) \\ 
&= \underbrace{4h^{-2}\sin^2(\pi kh/2)}_{\lambda_k}\underbrace{\sin(\pi jkh)}_{e^k_j} = \lambda_k e^k_j
\end{align*}
mit folgenden Theoremen:
\begin{align*}
\text{\scriptsize{(1):}}& \qquad \sin(x+y) = \sin(x)\cos(y) + \cos(x)\sin(y) \\ 
\text{\scriptsize{(2):}}& \qquad \sin(-x) = -\sin(x) \text{ und } \cos(-x) = \cos(x)\\ 
\text{\scriptsize{(3):}}& \qquad 2-2\cos(x) = 4\sin^2(x/2) 
\end{align*}
\section*{T13. Hauptachsentransformation}


\end{document}