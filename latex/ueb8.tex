\documentclass[11pt]{article}
\renewcommand{\baselinestretch}{1.05}
\usepackage{amsmath,amsthm,verbatim,amssymb,amsfonts,amscd, graphicx}
\usepackage{graphics, multirow}
\topmargin0.0cm
\headheight0.0cm
\headsep0.0cm
\oddsidemargin0.0cm
\textheight23.0cm
\textwidth16.5cm
\footskip1.0cm
\parindent0cm
\theoremstyle{plain}
\newtheorem{theorem}{Theorem}
\newtheorem{corollary}{Corollary}
\newtheorem{lemma}{Lemma}
\newtheorem{proposition}{Proposition}
\newtheorem*{surfacecor}{Corollary 1}
\newtheorem{conjecture}{Conjecture}
\newtheorem{question}{Question}
\theoremstyle{definition}
\newtheorem{definition}{Definition}
\let\mbb\boldsymbol
\renewcommand\boldsymbol{\mbb}
\renewcommand{\a}{\"{a}}
\renewcommand{\o}{\"{o}}
\renewcommand{\u}{\"{u}}
\newcommand{\beequal}{\mathop{=}\limits^!}
\newcommand{\equalstar}{\mathop{=}\limits^{(1)}}
\newcommand{\equalsstar}{\mathop{=}\limits^{(2)}}
\newcommand{\equalssstar}{\mathop{=}\limits^{(3)}}
\newcommand{\rayx}{\Lambda_A(\mbb{x})}
\newcommand{\rayv}{\Lambda_A(\mbb{v})}
\newcommand{\dray}{\nabla \Lambda_A(\mbb{x})}
\newcommand{\nat}{\mathbb{N}}
\newcommand{\real}{\mathbb{R}}
\newcommand{\reall}[2]{\mathbb{R}^{(#1 \times #2)}}

\begin{document}

\title{Numerische Mathematik f\u r Ingenieure (SS 14) - \"{U}bung 8}
\author{Merikan Koyun \& Julian Andrej}
\maketitle

\section*{T14. Eigenwerte und orthogonale Iteration}
Es seien $n, k \in \nat, n>k$ und $A \in \reall{n}{n}, Q \in \reall{n}{k}$ orthogonal. Es sei $\Lambda := Q^*AQ$ mit Eigenwert $\lambda$ und zugeh\o rigem Eigenvektor $x \in \real^k$, sodass $\Lambda x = \lambda x$. Es sei $y:=Qx$.
Zu zeigen:
\begin{equation}
\frac{\Vert Ay-\lambda y\Vert_2}{\Vert y \Vert_2} \leq \Vert Q\Lambda - AQ \Vert_2
\end{equation}

Es gilt $\Lambda \in \reall{k}{k}$. Wir beginnen zun\a chst mit der linken Seite der Ungleichung. Da $Q$ orthogonal ist, gilt $\Vert Qx \Vert_2 = \Vert x \Vert_2$ f\u r alle $x \in \real^k$, woraus folgt:

\begin{align*}
\frac{\Vert Ay-\lambda y\Vert_2}{\Vert y \Vert_2} = \frac{\Vert Ay-\lambda y\Vert_2}{\Vert Qx \Vert_2} = \frac{\Vert Ay-\lambda y\Vert_2}{\Vert x \Vert_2}
\end{align*}

Es folgt:
\begin{align*}
\frac{\Vert Ay-\lambda y\Vert_2}{\Vert x \Vert_2} &\leq \Vert Q\Lambda - AQ \Vert_2 \\
&\leq \Vert Q (Q^*AQ)-AQ \Vert_2 \\ 
&\leq \Vert (Q Q^*)AQ-AQ \Vert_2 \\
&\leq \Vert AQ-AQ \Vert_2 \\
&\leq \Vert 0 \Vert_2 \\
&\leq 0 \\
\end{align*}

Es folgt somit, dass $\frac{\Vert Ay-\lambda y\Vert_2}{\Vert x \Vert_2} = 0$ sein muss. Der Nenner darf nicht Null sein, somit muss gelten $\Vert Ay-\lambda y\Vert_2 = 0$. Mit der Definitheit der euklidischen Norm folgt $ Ay-\lambda y = 0$. Mit $y=Qx$ folgt:
\begin{equation}
AQx-\lambda Qx=0Q
\end{equation}


\section*{T15. Wilkinson-Shift}
Gegeben:
\begin{equation}
S= 
\begin{pmatrix}
a_{n-1,n-1}^{(m)} & a_{n-1,n}^{(m)} \\
a_{n,n-1}^{(m)} & a_{n,n}^{(m)} 
\end{pmatrix}
\end{equation}

Wir suchen die Nullstellen des charakteristischen Polynoms:
\begin{align}
p_S(\lambda) = |\lambda I-S| \\
&= \left| 
\begin{pmatrix}
a_{n-1,n-1}^{(m)} & a_{n-1,n}^{(m)} \\
a_{n,n-1}^{(m)} & a_{n,n}^{(m)} 
\end{pmatrix}
\right| \\
&= (\lambda - a_{n-1,n-1}^{(m)})(\lambda - a_{n,n}^{(m)}) - |a_{n-1,n}^{(m)}|^2
\end{align}

Mit Hilfe des Hinweises
\begin{equation}
m:=\frac{a_{n-1,n-1}^{(m)}+a_{n,n}^{(m)}}{2} \qquad d:=\frac{a_{n-1,n-1}^{(m)}-a_{n,n}^{(m)}}{2}
\end{equation}

ergibt sich:
\begin{align}
p_S(\lambda) &= (\lambda -m -d)(\lambda - m + d) - |a_{n-1,n}^{(m)}|^2\\
&= (\lambda - m)^2 - d^2 - |a_{n-1,n}^{(m)}|^2
\end{align}

und damit ergibt sich f\u r die Eigenwerte, nach Umstellen:
\begin{equation}
\lambda = m \pm \sqrt{d^2 + |a_{n-1,n}^{(m)}|^2}
\end{equation}

\end{document}











