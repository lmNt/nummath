\documentclass[11pt]{article}
\renewcommand{\baselinestretch}{1.05}
\usepackage{amsmath,amsthm,verbatim,amssymb,amsfonts,amscd, graphicx}
\usepackage{graphics, multirow}
\topmargin0.0cm
\headheight0.0cm
\headsep0.0cm
\oddsidemargin0.0cm
\textheight23.0cm
\textwidth16.5cm
\footskip1.0cm
\parindent0pt
\theoremstyle{plain}
\newtheorem{theorem}{Theorem}
\newtheorem{corollary}{Corollary}
\newtheorem{lemma}{Lemma}
\newtheorem{proposition}{Proposition}
\newtheorem*{surfacecor}{Corollary 1}
\newtheorem{conjecture}{Conjecture}
\newtheorem{question}{Question}
\theoremstyle{definition}
\newtheorem{definition}{Definition}
\let\mbb\boldsymbol
\renewcommand\boldsymbol{\mbb}
\renewcommand{\a}{\"{a}}
\renewcommand{\o}{\"{o}}
\renewcommand{\u}{\"{u}}
\newcommand{\beequal}{\mathop{=}\limits^!}

\begin{document}

\title{Numerische Mathematik f\u r Ingenieure (SS 14) - \"{U}bung 2}
\author{Merikan Koyun \& Julian Andrej}
\maketitle

\section*{T2. LR-Zerlegung an einem Beispiel}
\begin{itemize}
\item[a)]
Die Berechnung der einzelnen Eintr\a ge der $L$ und $R$ Matrix erfolgt \u ber die Matrixmultiplikation.

\begin{align}
\begin{array}{cccc|cccc|}
\cline{5-8}
 &  &  &         & 2  & -1 & -3 & 3   \\
 &  &  &         & 0  &  2 &  3 & -5  \\ 
 &  &  &         & 0  &  0 &  2 &  7  \\
 &  &  &         & 0  &  0 &  0 & -46 \\ \hline
\multicolumn{1}{|c}  1 & 0 & 0 & 0     & 2  & -1 & -3 & 3 \\
\multicolumn{1}{|c}  2 & 1 & 0 & 0    & 4  & 0 & -3 & 1   \\
\multicolumn{1}{|c}  3 & 2 & 1 & 0    & 6  & 1 & -1 & 6   \\
\multicolumn{1}{|c} -1 & -3 & 5 & 1  & -2  & -5 & 4 & 1   \\ \hline
\end{array}
\end{align}

$r_{1j} = a_{1j}, \quad \forall j\in[1,4]$
\begin{align}
1 \cdot r_{11}  &\beequal 2 \rightarrow r_{11}=2  \\ 
l_{21} r_{11}  &\beequal 4  \rightarrow l_{21}=2  \\
l_{31} r_{11}  &\beequal 6  \rightarrow l_{31}=3  \\
l_{41} r_{11}  &\beequal -2 \rightarrow l_{41}=-1 \\
2 \cdot -1 + 1 \cdot r_{22} &\beequal 0 \rightarrow r_{22} = 2 \\
2 \cdot -3 + 1 \cdot r_{23} &\beequal -3 \rightarrow r_{23} = 3 \\
2 \cdot 3  + 1 \cdot r_{24} &\beequal 1  \rightarrow r_{24} = -5 \\
3 \cdot -1 + l_{32} \cdot 2 &\beequal 1  \rightarrow l_{32} = 2 \\
-1 \cdot -1 + l_{42} \cdot 2 &\beequal -5 \rightarrow l_{42} = -3 \\
3 \cdot -3 + 2 \cdot 3 + 1 \cdot r_{33} &\beequal -1 \rightarrow r_{33} = 2 \\
3 \cdot 3 + 2 \cdot -5 + 1 \cdot r_{34} &\beequal 5 \rightarrow r_{34} = 7 \\
-1 \cdot -3 + -3 \cdot 3 + l_{43} \cdot 2 &\beequal 4 \rightarrow l_{43} = 5 \\
-1 \cdot 3 + -3 \cdot -5 + 5 \cdot 7 + 1 \cdot r_{44} &\beequal 1 \rightarrow r_{44}=-46
\end{align}


\item[b)]
Gesucht wird $x$ in $Ax=b$, f\u r
\begin{equation}
b = 
\begin{pmatrix}
1 \\ -8 \\ -16 \\ -12
\end{pmatrix}
\end{equation}


\textbf{Vorw\a rtseinsetzen}
Es gilt:
\begin{equation}
Ly=b
\end{equation}
In Matrixschreibweise:
\begin{equation}
\begin{pmatrix}
1 & 0 & 0 & 0 \\
2 & 1 & 0 & 0 \\
3 & 2 & 1 & 0 \\
-1& -3& 5 & 1
\end{pmatrix}
\begin{pmatrix}
y_1 \\ y_2 \\ y_3 \\ y_4
\end{pmatrix}
=
\begin{pmatrix}
1 \\ -8 \\ -16 \\ -12
\end{pmatrix}
\end{equation}

Forw\a rtseinsetzen ergibt folgende Berechnungen:
\begin{align}
1 \cdot y_1 = 1   &\rightarrow y_1 = 1 \\
2 + y_2 = -8      &\rightarrow y_2=-10 \\
3-20+y_3 =-16     &\rightarrow y_3 = 1 \\
-1+30+5+y_4 = -12 &\rightarrow y_4 = -46 \\
\end{align}
 
Durch R\u ckw\a rtseinsetzen kann $x$ \u ber die Beziehung $y=Rx$ berechnet werden:
\begin{equation}
\begin{pmatrix}
2 & -1 & -3 & 3 \\
0 & 2 & 3 & -5 \\
0 & 0 & 2 & 7 \\
0 & 0 & 0 & -46
\end{pmatrix}
\begin{pmatrix}
x_1 \\ x_2 \\ x_3 \\ x_4
\end{pmatrix}
=
\begin{pmatrix}
1 \\ -10 \\ 1 \\ -46
\end{pmatrix}
\end{equation}
R\u ckw\a rtseinsetzen ergibt folgende Berechnungen:
\begin{align}
1 \cdot x_4 = 1   &\rightarrow x_4 = 1 \\
2x_3 + 7 = -1      &\rightarrow x_3=-3 \\
2x_2-9-5 =-10     &\rightarrow x_2 = 2 \\
2x_1 -2 + 9 +3 = 1 &\rightarrow x_1 = -4.5
\end{align} 
\end{itemize}

\section*{T3. Hauptuntermatrizen und LR-Zerlegung}
Gegeben sei eine Matrix $\mbb{A}\in \mathbb{R}^{n \times n}$. Die Hauptuntermatrix $H_m$ ist regul\a r $\forall m \in \{1,...,n\}$. Zeige, dass $\mbb{A}$ eine LR-Zerlegung besitzt.\\

\textbf{Induktionsanfang:} $n = 1$ \\
Es folgt zwangsweise $l_{11} = 1$ und $r_{11} = a_{11}$.\\

\textbf{Induktionsvoraussetzung:} \\
Die Behauptung gilt f\u r alle Matrizen $\mathbb{R}^{n-1 \times n-1}$\\

\textbf{Induktionsschritt:} $n > 1$ \\
Es gilt aus der Bedigung dass alle $H_m$ regul\a r sind. $\mbb{A}$ wird in geeignete Bl\o cke zerlegt:
\begin{equation}
\mbb{A}=
\left(
\begin{array}{ccc|c}
a_{1,1} & \cdots & a_{1,n-1} 	 & a_{1,n} \\
\vdots  & \ddots & \vdots    	 & \vdots  \\
a_{n-1,1} & \cdots & a_{n-1,n-1} & a_{n-1,n} \\\hline
a_{n,1} & \cdots & a_{n,n-1} 	 & a_{n,n} 
\end{array}
\right)
\end{equation}
wobei der obere linke Block
\begin{equation}
\mbb{A}_{**} = 
\begin{pmatrix}
a_{1,1} & \cdots & a_{1,n-1}    \\
\vdots  & \ddots & \vdots       \\
a_{n-1,1} & \cdots & a_{n-1,n-1}\\
\end{pmatrix}
\end{equation}
die obere Hauptuntermatrix $H_{n-1}$ ist. Die verbleibenden Bl\o cke werden wie folgt, analog zur Vorlesung, bezeichnet:
\begin{align}
\mbb{A}_{n*} =
\begin{pmatrix}
a_{n,1} & \cdots & a_{n,n-1}
\end{pmatrix}\\
\mbb{A}_{*n} = 
\begin{pmatrix}
a_{1,n} \\
\vdots  \\
a_{n-1,n} 
\end{pmatrix}
\end{align}

Laut Voraussetzung sind die Hauptuntermatrizen $H_m$ regul\a r, also $\det(H_m) \neq 0$. Da laut IB $\mbb{A}_{**} = \mbb{L}_{**}\mbb{R}_{**}$, gilt 

\begin{equation}
\det(\mbb{A}_{**}) = \det(\mbb{L}_{**}\mbb{R}_{**}) = \det(\mbb{L}_{**}) \det(\mbb{R}_{**}) 
\end{equation}

und somit muss auch $\det(\mbb{L}_{**}) \neq 0$ und $\det (\mbb{R}_{**}) \neq 0$ gelten und damit sind $\mbb{L}_{**}$ und $\mbb{R}_{**}$ ebenfalls regul\a r.

Analog zu Abschnitt 2.4 in der Vorlesung gilt:
\begin{equation}
\mbb{L} = 
\begin{pmatrix}
\mbb{L}_{**} & 		\\
\mbb{L}_{n*} & 1 \\
\end{pmatrix}
\text{ und }
\mbb{R} = 
\begin{pmatrix}
\mbb{R}_{**} & \mbb{R}_{*n} \\
& r_{nn}
\end{pmatrix}
\end{equation}
wobei $r_{nn}$ analog zu T2) berechnet wird, da $l_{11}=1$:
\begin{equation}
r_{nn} = a_{nn} - \mbb{L}_{n*}\mbb{R}_{*n}
\end{equation}
Ausserdem gilt f\u r die Blockmatrizen:
\begin{equation}
\mbb{A}_{n*}=\mbb{L}_{n*}\mbb{R}_{**}
\text{ bzw. } 
\mbb{A}_{*n}=\mbb{L}_{**}\mbb{R}_{*n}
\end{equation}


Mit Hilfe vorangegangener Gleichungen folgt also:
\begin{equation}
LR = 
\begin{pmatrix}
\mbb{L}_{**}\mbb{R}_{**} & \mbb{L}_{**}\mbb{R}_{*n} \\
\mbb{L}_{n*}\mbb{R}_{**} & \mbb{L}_{n*}\mbb{R}_{*n}+l_{nn}r_{nn}
\end{pmatrix}
=
\begin{pmatrix}
\mbb{A}_{**} & \mbb{A}_{*n} \\
\mbb{A}_{n*} & a_{nn}   \\
\end{pmatrix}
\end{equation}
somit l\a sst sich direkt erkennen, dass $\mbb{L}\mbb{R}=\mbb{A}$ gilt und somit sind $\mbb{L}$ und $\mbb{R}$ eine LR-Zerlegung von $\mbb{A}$. $_\blacksquare$
 
 
 
 
 
 
 
 
 
 
 
 
 
 
 
 
 
 
 
\end{document}