\documentclass[11pt]{article}
\renewcommand{\baselinestretch}{1.05}
\usepackage{amsmath,amsthm,verbatim,amssymb,amsfonts,amscd, graphicx}
\usepackage{graphics, multirow}
\topmargin0.0cm
\headheight0.0cm
\headsep0.0cm
\oddsidemargin0.0cm
\textheight23.0cm
\textwidth16.5cm
\footskip1.0cm
\parindent0cm
\theoremstyle{plain}
\newtheorem{theorem}{Theorem}
\newtheorem{corollary}{Corollary}
\newtheorem{lemma}{Lemma}
\newtheorem{proposition}{Proposition}
\newtheorem*{surfacecor}{Corollary 1}
\newtheorem{conjecture}{Conjecture}
\newtheorem{question}{Question}
\theoremstyle{definition}
\newtheorem{definition}{Definition}
\let\mbb\boldsymbol
\renewcommand\boldsymbol{\mbb}
\renewcommand{\a}{\"{a}}
\renewcommand{\o}{\"{o}}
\renewcommand{\u}{\"{u}}
\newcommand{\beequal}{\mathop{=}\limits^!}
\newcommand{\equalstar}{\mathop{=}\limits^{(1)}}
\newcommand{\equalsstar}{\mathop{=}\limits^{(2)}}
\newcommand{\equalssstar}{\mathop{=}\limits^{(3)}}
\newcommand{\rayx}{\Lambda_A(\mbb{x})}
\newcommand{\rayv}{\Lambda_A(\mbb{v})}
\newcommand{\dray}{\nabla \Lambda_A(\mbb{x})}
\newcommand{\nat}{\mathbb{N}}
\newcommand{\real}{\mathbb{R}}
\newcommand{\reall}[2]{\mathbb{R}^{(#1 \times #2)}}

\begin{document}

\title{Numerische Mathematik f\u r Ingenieure (SS 14) - \"{U}bung 9}
\author{Merikan Koyun \& Julian Andrej}
\maketitle

\section*{T16. Lagrange Polynome und Monome}
Wir betrachten zun\a chst den allgemeinen Fall:
\begin{equation}
p(x) = \sum_{i=0}^m x_i^k l_i(x)
\end{equation}
Der Grad des Polynoms ist also h\o chstens $m$ und es gilt $p(x_s)=x_s^k$ f\u r $k=0,...,m$.
Weiterhin hat das neue Polynom $q(x)=p(x)-x^k$ die $m+1$ Nullstellen $x_0, ..., x_m$.\\

$k=0$:\\
Es gilt somit $q(x) = p(x) - x^0 = p(x) - 1$. $q(x)$ ist ein Polynom vom Grad $m$ mit $m+1$ Nullstellen $x_0, ..., x_m$ und ist daher das Nullpolynom, sodass $p(x)=x^0=1$ f\u r alle $x$. Dann gilt trivialerweise auch f\u r $p(0)=1$. \vspace{0.3cm}

$k=1,...,n$:\\
$q(x)=p(x)-x^k$ ist ebenfalls ein Polynom vom Grad $m$ und hat die $m+1$ Nullstellen $x_0, ..., x_m$. Es ist ebenfalls das Nullpolynom, sodass gilt $p(x)=x^k$ f\u r alle x, sodass $p(0)=0$ gilt. \vspace{0.3cm}

$k=m+1$:\\
$q(x)=p(x)-x^{m+1}$ ist ein Polynom vom Grad $m+1$ mit $m+1$ Nullstellen $x_0, ..., x_m$. Der Koeffizient $a_{m+1}=-1$ (Leitkoeffizient).
Es gilt also $q(x) = p(x) - x^{m+1} = - \prod_{i=0}^m (x-x_i)$. Umgestellt ergibt sich
\begin{equation}
p(x)= \left(- \prod_{i=0}^m (x-x_i)\right) + x^{m+1}
\end{equation}
Es folgt:
\begin{align}
p(0) &= -\prod_{i=0}^m -x_i \\
&= (-1)^m\prod_{i=0}^m x_i
\end{align}


\section*{T17. Interpolationsfehler}
Die Funktion $f(x)=e^{-x^2}$ soll an \a quidistanten St\u tzstellen $x_i = ih, i=0,1,2...$ auf dem Intervall $[0,1]$ tabelliert werden. Es ist die Schrittweite $h$ gesucht, die bei linearer Interpolation einen Interpolationsfehler kleiner $10^{-6}$ erzeugt.\vspace{0.3cm}

F\u r den Fehler gilt:
\begin{equation}
e(x) = f(x)-p(x) = \frac{f''(\eta)}{2}(x-x_i)(x-x_{i+1})
\end{equation}

Zur Absch\a tzung des maximalen Fehlers bedienen wir uns der Maximumsnorm
\begin{equation}
\Vert g \Vert_{\infty, [a,b]} = \max\{|g(x)| \, : \, x\in [a,b]\}
\end{equation}
sodass gilt:
\begin{equation}
\Vert f(x)-p(x)\Vert_{\infty, [a,b]} \leq \frac{\Vert f''(\eta) \Vert_{\infty, [a,b]}}{2}\Vert(x-x_i)(x-x_{i+1})\Vert_{\infty, [a,b]}
\end{equation}

Wir berechnen zun\a chst die Maximumsnorm von $\Vert(x-x_i)(x-x_{i+1})\Vert_{\infty, [x_i,x_{i+1}]}$.
\begin{align}
\Vert(x-x_i)(x-x_{i+1})\Vert_{\infty, [x_i,x_{i+1}]} &= \max_{[x_i,x_{i+1}]} |(x-x_i)(x-x_{i+1})| \\
&= \frac{(x_{i+1}-x_i)^2}{4}\\
&= \frac{(ih+h-ih)^2}{4} \\ 
&= \frac{h^2}{4}
\end{align}

Es gilt weiterhin:
\begin{equation}
f''(x) = e^{-x^2}(4x^2-2)
\end{equation}
und
\begin{equation}
\max_{[0,1]} |e^{-x^2}(4x^2-2)| = 2
\end{equation}

Wir k\o nnen den Fehler nun absch\a tzen mit:
\begin{equation}
e(x) \leq \frac{2h^2}{8} = \frac{h^2}{4}
\end{equation}

Der Fehler soll kleiner als $10^{-6}$ sein, somit ergibt sich:
\begin{equation}
h \leq \sqrt{4\cdot 10^{-6}} = 0.002
\end{equation}
\end{document}











