\documentclass[11pt]{article}
\renewcommand{\baselinestretch}{1.05}
\usepackage{amsmath,amsthm,verbatim,amssymb,amsfonts,amscd, graphicx}
\usepackage{graphics, multirow}
\topmargin0.0cm
\headheight0.0cm
\headsep0.0cm
\oddsidemargin0.0cm
\textheight23.0cm
\textwidth16.5cm
\footskip1.0cm
\parindent0cm
\theoremstyle{plain}
\newtheorem{theorem}{Theorem}
\newtheorem{corollary}{Corollary}
\newtheorem{lemma}{Lemma}
\newtheorem{proposition}{Proposition}
\newtheorem*{surfacecor}{Corollary 1}
\newtheorem{conjecture}{Conjecture}
\newtheorem{question}{Question}
\theoremstyle{definition}
\newtheorem{definition}{Definition}
\let\mbb\boldsymbol
\renewcommand\boldsymbol{\mbb}
\renewcommand{\a}{\"{a}}
\renewcommand{\o}{\"{o}}
\renewcommand{\u}{\"{u}}
\newcommand{\beequal}{\mathop{=}\limits^!}
\newcommand{\equalstar}{\mathop{=}\limits^{(1)}}
\newcommand{\equalsstar}{\mathop{=}\limits^{(2)}}
\newcommand{\equalssstar}{\mathop{=}\limits^{(3)}}
\newcommand{\rayx}{\Lambda_A(\mbb{x})}
\newcommand{\rayv}{\Lambda_A(\mbb{v})}
\newcommand{\dray}{\nabla \Lambda_A(\mbb{x})}
\newcommand{\nat}{\mathbb{N}}
\newcommand{\real}{\mathbb{R}}
\newcommand{\reall}[2]{\mathbb{R}^{(#1 \times #2)}}

\begin{document}

\title{Numerische Mathematik f\u r Ingenieure (SS 14) - \"{U}bung 9}
\author{Merikan Koyun \& Julian Andrej}
\maketitle

\section*{T16. Lagrange Polynome und Monome}

Fall $k=0$:\\
Es gilt:
\begin{equation}
L(x)=\sum_{i=0}^m l_i(x) = 1
\end{equation}

Wir w\a hlen ein Polynom
\begin{equation}
P(x) = L(x)-1 = \sum_{i=0}^m l_i(x) = \sum_{i=0}^m \prod_{j=0, j \neq i}^m \frac{x-x_j}{x_i-x_j}
\end{equation} 
Es l\a sst sich erkennen, dass
\begin{equation}
P(x_i)=0 \quad \forall i = 0,...,m
\end{equation}
Dieses Polynom ist demnach vom Grad $n$ und besitzt $n+1$ Nullstellen. Daraus folgt, dass $P(x)$ das Nullpolynom sein muss.

Es folgt, dass $L(x)$ das konstante Polynom 1 und es folgt:
\begin{equation}
\sum_{i=0}^m l_i(x) = 1
\end{equation}

\section*{T17. Interpolationsfehler}
Die Funktion $f(x)=e^{-x^2}$ soll an \a quidistanten St\u tzstellen $x_i = ih, i=0,1,2...$ auf dem Intervall $[0,1]$ tabelliert werden. Es ist die Schrittweite $h$ gesucht, die bei linearer Interpolation einen Interpolationsfehler kleiner $10^{-6}$ erzeugt.\vspace{0.3cm}

F\u r den Fehler gilt:
\begin{equation}
e(x) = f(x)-p(x) = \frac{f''(\eta)}{2}(x-x_i)(x-x_{i+1})
\end{equation}

Zur Absch\a tzung des maximalen Fehlers bedienen wir uns der Maximumsnorm
\begin{equation}
\Vert g \Vert_{\infty, [a,b]} = \max\{|g(x)| \, : \, x\in [a,b]\}
\end{equation}
sodass gilt:
\begin{equation}
\Vert f(x)-p(x)\Vert_{\infty, [a,b]} \leq \frac{\Vert f''(\eta) \Vert_{\infty, [a,b]}}{2}\Vert(x-x_i)(x-x_{i+1})\Vert_{\infty, [a,b]}
\end{equation}

Wir berechnen zun\a chst die Maximumsnorm von $\Vert(x-x_i)(x-x_{i+1})\Vert_{\infty, [a,b]}$.
\begin{align}
\Vert(x-x_i)(x-x_{i+1})\Vert_{\infty, [x_i,x_{i+1}]} &= \max_{[x_i,x_{i+1}]} |(x-x_i)(x-x_{i+1})| \\
&= \frac{(x_{i+1}-x_i)^2}{4}\\
&= \frac{(ih+h-ih)^2}{4} \\ 
&= \frac{h^2}{4}
\end{align}

Es gilt weiterhin:
\begin{equation}
f''(x) = e^{-x^2}(4x^2-2)
\end{equation}
und
\begin{equation}
\max_{[0,1]} |e^{-x^2}(4x^2-2)| = 2
\end{equation}

Wir k\o nnen den Fehler nun absch\a tzen mit:
\begin{equation}
e(x) \leq \frac{2h^2}{8} = \frac{h^2}{4}
\end{equation}

Der Fehler soll kleiner als $10^{-6}$ sein, somit ergibt sich:
\begin{equation}
h \leq \sqrt{4\cdot 10^{-6}} = 0.002
\end{equation}
\end{document}











